\section{Einleitung}\label{sec:einleitung}

\textit{Ist eine KI dazu fähig moralische Entscheidungen zu treffen?} \\
Viele Menschen beschäftigen sich mit dieser Frage, denn sie ermöglicht eine
ganz neue Ebene der Interaktion mit den uns bekannten KIs.
Dieses Thema hat nicht nur gesellschaftliche Relevanz, sondern wird auch
aktiv erforscht.
Jedes Jahr findet eine AIES\footnote{Conference on AI, Ethics and Society} Konferenz statt, wo viele
Forscher ihre Studien zu diesem Thema veröffentlichen können.
Aber \texit{was ist eine moralische KI}, \textit{was wird gerade daran erforscht} und \textit{wie sieht die Zukunft moralischer KIs aus}.

Bei vielen Studien kann man schnell den Überblick verlieren und sehr schlecht einen Einstieg in das Thema bekommen.
Eine Übersicht hilft hier, den Einstieg in das Thema zu erleichtern und eine kurze Zusammenfassung über
den aktuellen Forschungsstand zu liefern.
Diese Seminararbeit beschäftigt sich mit den genannten Fragen und gibt einen Überblick über die aktuelle Forschung und die Zukunftsaussichten.

Am Anfang geht sie in die Ethik und Moral um ein Grundverständnis davon aufzu-bauen.
Anschließend wird erklärt, was die moralische KI ist und wozu sie nützlich sein könnte.
Im Hauptteil, werden ein paar der aktuellen Studien und die aktuelle Forschung vorgestellt.
Darunter, wie KI aus Texten den moralischen Wert von Wörtern erlernen kann und wie verschiedene moralische
KIs sich gegenseitig beeinflussen könnten.
Am Schluss gibt die Seminararbeit noch einen kleinen Einblick in mögliche Prognosen, wohin sich
moralische KIs in Zukunft weiterentwickeln könnten.
Hier werden verschiedene Meinungen von Forschern gezeigt, denn es ist nicht möglich klar eine
Zukunft bei einem so neuen Thema vorauszusagen.


