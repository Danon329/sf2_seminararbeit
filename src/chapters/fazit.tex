\section{Fazit}\label{sec::fazit}

Die moralische KI ist in vielen Gebieten noch unerforscht.
Von Uneinigkeiten in der Definition, über ihren Nutzen in der Gesellschaft und der Technik, bis zu der
Forschung und Entwicklung.
Auf allen diesen Ebenen gibt es Diskussionen und verschiedene Meinungen die vehement vertreten werden.
Jedoch ist ein großer Fortschritt zu beobachten.
Es sprechen viele technische und gesellschaftliche Vorteile für die Benutzung und Entwicklung der
\textit{Artificial Morality} (siehe ~\ref{subsec::artificial morality}).
Sie wird beim autonomen Fahren unerlässlich sein, aber auch beim Pflegekräftemangel, der jetzt schon
existiert und in Zukunft immer weiter ansteigen wird.
Es gibt Gründe und Motivationen um im ganzen Gebiet der moralischen KI weiter zu entwickeln zu forschen und dieses
Voranzubringen.

Es gibt nicht nur Motivationen, um weiterzugehen.
Die aktuellen Studien wie viel man mit moralischen KIs machen kann und wie tiefgründig das Thema ist.
Sei es nun, dass Forschende der TU Darmstadt es geschafft haben einer KI beizubringen aus Texten, den moralischen
Wert von Wörtern herauszulesen, oder Forschende der Universität London, die ein System entwickelt haben, welches uns
ermöglicht das Verhalten von moralischen KIs in Gruppen mit anderen moralischen Ausrichtung zu ermitteln und vorherzusagen.
Allein diese beiden Studien bringen uns immer näher an die mögliche Benutzung in den Feldern der Autonomie oder der Medizin.

Aber auch die Zukunft ist nicht wenig diskutiert.
Viele verschiedene Experten sehen die Zukunft mit moralischen KIs in einem guten Licht, dass die Menschheit
auch viel von der KI im Bereich Moral lernen wird.
Die apokalyptische Seite ist auch vertreten, die besagt, dass die KIs streng reguliert sein müssen, wenn wir eine
existenzielle Bedrohung verhindern wollen.

Moralische KI ist heiß diskutiert, jedoch wird immer weiter an ihr geforscht und die Zukunft verspricht
Interessant zu bleiben.