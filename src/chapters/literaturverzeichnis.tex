\section{Literatur- und Quellenverzeichnis}\label{sec::literaturverzeichnis}

Altmann J, et al.
\textit{Gratwanderung Künstliche Intelligenz: interdisziplinäre Perspektiven auf das Verhältnis von Mensch und KI}.
1st ed.
(Konz B, Ostmeyer K-H, Scholz M, eds.). Stuttgart: Verlag W. Kohlhammer; 2023. \\
\newline
Funk M.
\textit{Roboter- und KI-Ethik: eine methodische Einführung}.
Wiesbaden; [Heidelberg]: Springer Vieweg; 2022. \\
\newline
Meyer, L.: Art.
"Pflichtenethik" (Version 1.0 vom 12.10.2017), in: Ethik-Lexikon, verfügbar unter:
\href{https://www.ethik-lexikon.de/lexikon/pflichtenethik}{https://www.ethik-lexikon.de/lexikon/pflichtenethik} \\
\newline
Misselhorn C. Artificial Moral Agents:
Conceptual Issues and Ethical Controversy.
In: Voeneky S, Kellmeyer P, Mueller O, Burgard W, eds.
\textit{The Cambridge Handbook of Responsible Artificial Intelligence: Interdisciplinary Perspectives}.
Cambridge Law Handbooks.
Cambridge University Press; 2022:31-49. \\
\newline
Misselhorn, Catrin.
\textit{Maschinenethik und "Artificial Morality"};
[\href{https://www.bpb.de/shop/zeitschriften/apuz/263684/maschinenethik-und-artificial-morality/}{https://www.bpb.de/shop/zeit-schriften/apuz/263684/maschinenethik-und-artificial-morality/};
letzter Abruf: 07.04.2025] \\
\newline
Schedel, T.: Art.
"Utilitarismus" (Version 1.0 vom 12.11.2018), in: Ethik-Lexikon, verfügbar
unter: \href{https://www.ethik-lexikon.de/lexikon/utilitarismus}{https://www.ethik-lexikon.de/lexikon/utilitarismus} \\
\newline
Schneider, Gerd und Toyka-Seid, Christiane:
\textit{Moral};
[\href{https://www.bpb.de/kurz-knapp/lexika/das-junge-politik-lexikon/320812/moral/}{https://www.bpb.de/kurz-knapp/-lexika/das-junge-politik-lexikon/320812/moral/};
letzter Abruf 05.04.25] \\





