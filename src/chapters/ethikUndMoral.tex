\section{Ethik und Moral}\label{sec:ethik und moral}

\subsection{Was ist Moral?}\label{subsec::moral}

Unter Moral versteht man die in einer Gesellschaft, allgemein anerkannten Werte und Regeln.
Diese Werte und Regeln werden durch ständiges Hinterfragen des eigenen Handelns aufrechterhalten.
Das Hinterfragen des eigenen moralischen Handelns beruht dabei auf Geboten, die solches Handeln vorschreiben,
wie zum Beispiel: Man soll nicht töten oder stehlen.
Solche Gebote dienen als Grundlage einer Gesellschaft und Religionen und können so auch auf
eine moralische KI angewendet werden (siehe ~\ref{subsec::artificial morality}). % Hier muss ein Href hin!!!!


\subsection{Ethische Grundlagen der KI}\label{subsec::ethische grundlagen}

Es gibt zwei größere Theorien, die Pflichtenethik (siehe ~\ref{subsubsec::pflichtenethik}) und den Utilitarismus (siehe ~\ref{subsubsec::utilitarismus}),
die ethisches und moralisches Handeln beschreiben.
Beide dieser Theorien können als Grundlage verwendet werden, um die Moral einer moralischen KI zu definieren und um zu verstehen, welche
dieser Theorien geeigneter für eine KI ist, müssen beide Theorien in ihren Grundaussagen betrachtet werden.

\subsubsection{Pflichtenethik}\label{subsubsec::pflichtenethik}

Primär setzt sich die Pflichtenethik mit der Frage auseinander: \("\)Was soll ich tun?\("\).
Diese Norm soll regulierend sein, deshalb wird sie Pflichtenethik genannt.
Es werden dabei zwei Pflichten unterschieden:
\textbf{ideales Handeln aus Pflicht} und \textbf{pflichtgemäßes Handeln}.
Beim idealen Handeln aus Pflicht, handelt eine Person zum Beispiel aus Wohltätigkeit,
hier wird oft von Moralität gesprochen, auf der
anderen Seite muss eine Person, beim pflichtgemäßen Handeln, nicht aus wohltätigen Motiven handeln.
Sei es nun ein Helfersyndrom (Eine Person wird glücklicher beim anderen Helfen) oder um der Gesellschaft zu gefallen,
kann dann nicht mehr vom idealen Handeln aus Pflicht gesprochen werden.
Pflicht und pflichtgemäßes Handeln sehen von außen immer gleich aus, deshalb ist laut L. Meyer die richtige
Einstellung entscheidend für das richtige Handeln.
\("\)Entscheidend für ein Handeln aus Pflicht ist die richtige Gesinnung,
die als guter Wille allein für die richtigen Motive einer Handlung garantieren kann.\("\)\footnote{ Meyer, L.: Art.
\textit{Pflichtenethik} (Version 1.0 vom 12.10.2017), in: Ethik-Lexikon, verfügbar unter:
https://www.ethik-lexikon.de/lexikon/pflichtenethik}
Die Moralität in der Pflichtenethik ist stark an die Selbstbeurteilung gebunden.


\subsubsection{Utilitarismus}\label{subsubsec::utilitarismus}

Der Utilitarismus beruht als Grundlage auf der Frage der Nützlichkeit.
Die Nützlichkeit im Utilitarismus wird allgemein als Maximierung der Freude und Minimierung von Leid angesehen.
Laut T. Schedel sollen die Folgen einer Handlung das größtmögliche Glück für eine größtmögliche Menge, der von der Handlung
betroffenen bewirken.\footnote{ vgl. Schedel, T.: Art.
\textit{Utilitarismus} (Version 1.0 vom 12.11.2018), in: Ethik-Lexikon, verfügbar
unter: https://www.ethik-lexikon.de/lexikon/utilitarismus}

Für die Moral, die als Grundlage die Nützlichkeit des Utilitarismus hat, gilt, dass die Handlung moralisch ist,
solange sie das gemeine Glück befördern und unmoralisch, wenn sie Unglück fördern.
\hypertarget{Trolley-Problem}
Das Trolley-Problem gibt hier ein gutes Beispiel für eine echte Umsetzung und Interpretation des Utilitarismus, der
vom Glück abweicht.
\("\)Man gebe den Betroffenen einen Wert und stelle die Weiche auf das Gleis der
Betroffenen mit dem geringeren Wert.\("\)\footnote{ Schedel, T.: Art. 2018}
Die hier vorliegende Zuweisung von Werten an verschiedene Teile einer Entscheidung, gibt mögliche Ansätze für die
Entscheidungslogik einer den Moralvorstellungen des Utilitarismus folgenden KI.