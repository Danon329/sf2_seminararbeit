\section{Die Zukunft}\label{sec::zukunft}

Es gibt viele verschiedene Wege, in die sich die moralische KI in Zukunft weiterentwickeln kann.
Obwohl weiterhin sehr viel an dem Erlernen von Moral als auch an den Auswirkungen einer
moralischen KI geforscht wird, ist die Prognose für die weitere Entwicklung von moralischer KI sehr spekulativ.

Für eine zukünftige Benutzung von diesen autonomen KIs, werden oftmals das Militär, die Krankenpflege
und die autonome Fortbewegung als Beispiele genannt. 
Leider ist es jedoch noch unklar wie weit das moralische Verständnis von KI weiter ausgebaut werden kann.
Es gibt positive und negative Prognosen, für moralische KIs und deren Entwicklung.\footnote{vgl. Martinho, A. et al. Perspectives about artificial moral agents; \textit{3.4 About future projections for AMAs}}

\subsection{Positive Prognose}\label{subsec::positive prognose}

Die positive Prognose besagt, dass moralische KIs unvermeidbar sind und damit auch
unersetzbar für den technologischen Fortschritt.\footnote{vgl. Martinho, A. et al. Perspectives about artificial moral agents; \textit{4.1 Perspective 1 machine ethics: the way forward}}

Nicht nur wird es für den technologischen Fortschritt unerlässlich sein, sondern könnte
durch die konstante weiterentwicklung auch unser Verständnis von Moral verändern.
Möglicherweise auch verbessern.\footnote{vgl. Martinho, A. et al. Perspectives about artificial moral agents; \textit{4.1.4 Future Projections}}

\subsection{Neutrale Prognose}\label{subsec::neutrale prognose}

Die neutrale Prognose sagt voraus, dass moralische KIs Menschen nicht in schwierigen moralischen
Situationen ersetzen werden.
Das liegt daran, dass Ethik als auch die menschliche Moral nicht algorithmisch aufgebaut ist.\footnote{vgl. Martinho, A. et al. Perspectives about artificial moral agents; \textit{4.2 Perspective 2 ethical verification: sage and sufficient}}

Selbst wenn die moralische KI menschliche Werte bekommen sollte, wird die KI keinen
Einfluss auf das menschliche Verständnis von Moral haben.\footnote{vgl. Martinho, A. et al. Perspectives about artificial moral agents; \textit{4.2.4 Future projections}}

\subsection{Negative Prognose}\label{subsec::negative prognose}

Eine moralische KI muss moralisch unsicher sein und dabei menschliche Werte beinhalten, ansonsten
könnten sie eine existenzielle Bedrohung für die Menschheit darstellen.
Es ist nicht genug, externe Einschränkungen und Überwachungen einzubauen, wie auch unethisches Verhalten
zu verbieten.\footnote{vgl. Martinho, A. et al. Perspectives about artificial moral agents; \textit{4.3 Perspective 3 morally uncertain machines: human values to avoid moral dystopia}}

Die bloße indifferenz für menschliche Werte, welches das menschliche Überleben beinhaltet, könnte
genug sein, dass moralische KIs eine existenzielle Bedrohung darstellen könnten.\footnote{vgl. Martinho, A. et al. Perspectives about artificial moral agents; \textit{4.3.4 Future projections}}